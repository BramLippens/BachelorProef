%%=============================================================================
%% Methodologie
%%=============================================================================

\chapter{\IfLanguageName{dutch}{Methodologie}{Methodology}}%
\label{ch:methodologie}

%% TODO: In dit hoofstuk geef je een korte toelichting over hoe je te werk bent
%% gegaan. Verdeel je onderzoek in grote fasen, en licht in elke fase toe wat
%% de doelstelling was, welke deliverables daar uit gekomen zijn, en welke
%% onderzoeksmethoden je daarbij toegepast hebt. Verantwoord waarom je
%% op deze manier te werk gegaan bent.
%% 
%% Voorbeelden van zulke fasen zijn: literatuurstudie, opstellen van een
%% requirements-analyse, opstellen long-list (bij vergelijkende studie),
%% selectie van geschikte tools (bij vergelijkende studie, "short-list"),
%% opzetten testopstelling/PoC, uitvoeren testen en verzamelen
%% van resultaten, analyse van resultaten, ...
%%
%% !!!!! LET OP !!!!!
%%
%% Het is uitdrukkelijk NIET de bedoeling dat je het grootste deel van de corpus
%% van je bachelorproef in dit hoofstuk verwerkt! Dit hoofdstuk is eerder een
%% kort overzicht van je plan van aanpak.
%%
%% Maak voor elke fase (behalve het literatuuronderzoek) een NIEUW HOOFDSTUK aan
%% en geef het een gepaste titel.

Het onderzoek is gestructureerd in drie opeenvolgende fasen, die nauwgezet op elkaar zullen volgen om een grondige en systematische analyse van het spel en zijn vereisten te waarborgen.

In de eerste fase zal uitgebreide informatie worden verzameld om te bepalen welke game-engines voldoen aan een uitgebreide reeks essentiële criteria, zoals vastgesteld volgens het \\MoSCoW-principe. Deze criteria zullen in overleg met AllPhi worden vastgesteld, met als doel een grondig begrip te krijgen van hun specifieke vereisten en voorkeuren met betrekking tot het spel. Gedurende een geschatte periode van 2 tot 3 weken zal deze fase zich richten op het verkennen van verschillende opties.

De tweede fase van het onderzoek zal zich richten op het verfijnen van de initiële lijst van game-engines tot een meer beheersbare selectie, waarbij de nadruk ligt op het identificeren van die engines die voldoen aan een groot aantal should- en could-have criteria voor het spel. Deze fase, gepland over een periode van 2 weken, zal een diepere analyse omvatten van de potentieel geschikte opties die voortkomen uit de eerste fase, waarbij aandacht wordt besteed aan aspecten zoals functionaliteit, gebruiksgemak en technische ondersteuning.

Ten slotte zal de derde fase van het onderzoek zich concentreren op de praktische uitvoering door middel van het ontwikkelen van een proof-of-concept met behulp van een zorgvuldig geselecteerde game-engine uit de kortere lijst. Dit proces zal in nauwe samenwerking met AllPhi worden uitgevoerd om ervoor te zorgen dat het eindproduct volledig voldoet aan hun specifieke behoeften en verwachtingen met betrekking tot het spel. Gezien de complexiteit en de omvang van deze fase wordt een aanzienlijke tijdsperiode van 6 tot 8 weken gereserveerd om een grondige en zorgvuldige ontwikkeling te garanderen.

