%%=============================================================================
%% Conclusie
%%=============================================================================

\chapter{Conclusie}%
\label{ch:conclusie}

% TODO: Trek een duidelijke conclusie, in de vorm van een antwoord op de
% onderzoeksvra(a)g(en). Wat was jouw bijdrage aan het onderzoeksdomein en
% hoe biedt dit meerwaarde aan het vakgebied/doelgroep? 
% Reflecteer kritisch over het resultaat. In Engelse teksten wordt deze sectie
% ``Discussion'' genoemd. Had je deze uitkomst verwacht? Zijn er zaken die nog
% niet duidelijk zijn?
% Heeft het onderzoek geleid tot nieuwe vragen die uitnodigen tot verder 
%onderzoek?
Om de onderzoeksvraag te beantwoorden: "Welke 2D game-engines zijn geschikt voor programmeurs van AllPhi met een C\# achtergrond, maar zonder game development-ervaring, om games te ontwikkelen voor evenementen?", biedt de ontwikkeling van de proof-of-concept waardevolle inzichten. Godot blijkt een toegankelijke instap voor het leren ontwikkelen van games, waarbij de ondersteuning voor C\# voldoende is gebleken voor het creëren van eenvoudige videogames.