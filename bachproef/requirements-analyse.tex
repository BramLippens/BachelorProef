%%=============================================================================
%% Methodologie
%%=============================================================================

\chapter{\IfLanguageName{dutch}{Requirements-analyse}{Requirements-analyse}}%
\label{ch:requirements-analyse}

%% TODO: In dit hoofstuk geef je een korte toelichting over hoe je te werk bent
%% gegaan. Verdeel je onderzoek in grote fasen, en licht in elke fase toe wat
%% de doelstelling was, welke deliverables daar uit gekomen zijn, en welke
%% onderzoeksmethoden je daarbij toegepast hebt. Verantwoord waarom je
%% op deze manier te werk gegaan bent.
%% 
%% Voorbeelden van zulke fasen zijn: literatuurstudie, opstellen van een
%% requirements-analyse, opstellen long-list (bij vergelijkende studie),
%% selectie van geschikte tools (bij vergelijkende studie, "short-list"),
%% opzetten testopstelling/PoC, uitvoeren testen en verzamelen
%% van resultaten, analyse van resultaten, ...
%%
%% !!!!! LET OP !!!!!
%%
%% Het is uitdrukkelijk NIET de bedoeling dat je het grootste deel van de corpus
%% van je bachelorproef in dit hoofstuk verwerkt! Dit hoofdstuk is eerder een
%% kort overzicht van je plan van aanpak.
%%
%% Maak voor elke fase (behalve het literatuuronderzoek) een NIEUW HOOFDSTUK aan
%% en geef het een gepaste titel.

In overleg met AllPhi zijn de volgende use cases geformuleerd. We hanteren het MoSCoW-principe om een overzicht te verkrijgen van hoe we de use cases zullen ordenen. MoSCoW staat voor Must-haves, Should-haves, Could-haves en Mustn't-haves.

\section{Functional requirements}
\subsection{Gameplay}
\begin{itemize}
    \item De speler moet een ruimteschip kunnen bewegen zowel horizontaal als verticaal. 
    \item Er moeten vijanden in “spawnen” in een formatie. 
    \item Vijanden moeten van links naar rechts en vice versa bewegen.
    \item Het ruimteschip moet kogels kunnen afvuren om zo vijanden te kunnen vermoorden.
    \item Als een vijand geraakt wordt moet deze weggaan en de speler een punt krijgen. 
    \item Als alle vijanden van een bepaald level dood zijn moet de speler naar het volgend level kunnen gaan.
    \item Er moet een moeilijkheidsgraad dynamisch toenemen met elk level zodat de vijanden moeilijker te verslaan worden. 
    \item De speler heeft een beperkt aantal levens. 
    \item Als een speler geraakt wordt door een vijand moet deze een leven verliezen.
    \item Als een speler al zijn levens verloren heeft moet het spel stoppen. 
\end{itemize}

\subsection{\IfLanguageName{dutch}{Puntensysteem}{Pointsystem}}
\begin{itemize}
    \item Hoe hoger de moeilijkheidsgraad hoe meer punten er verdiend worden bij het verslaan van vijanden.
    \item Het spel moet de scores bij kunnen houden van spelers. 
    \item Het spel moet een scorebord kunnen tonen met een ranglijst van de beste spelers. 
\end{itemize}

\subsection{Power-ups}
\begin{itemize}
    \item Bij het verslaan van een vijand kan het gebeuren dat deze een power-up laat vallen.
    \item De speler kan een power-up verzamelen die het wapen kan verbeteren.
    \item De speler kan extra levens opnemen zodat ze langer kunnen spelen.
\end{itemize}

\subsection{\IfLanguageName{dutch}{Spelmodi}{Gamemodes}}
\begin{itemize}
    \item Het spel heeft een single player modus.
    \item Het spel heeft een multiplayer modus.
\end{itemize}

\section{Non-Functional requirements}
\begin{itemize}
    \item Het spel moet een gebruiksvriendelijke gebruikersinterface hebben. 
    \item Het spel moet soepel kunnen draaien ookal zijn er veel vijanden op het scherm. 
    \item Er moeten geluidseffecten toegevoegd worden bij het spel.
\end{itemize}
