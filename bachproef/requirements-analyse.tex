%%=============================================================================
%% Methodologie
%%=============================================================================

\chapter{\IfLanguageName{dutch}{Requirements-analyse}{Requirements-analyse}}%
\label{ch:requirements-analyse}

%% TODO: In dit hoofstuk geef je een korte toelichting over hoe je te werk bent
%% gegaan. Verdeel je onderzoek in grote fasen, en licht in elke fase toe wat
%% de doelstelling was, welke deliverables daar uit gekomen zijn, en welke
%% onderzoeksmethoden je daarbij toegepast hebt. Verantwoord waarom je
%% op deze manier te werk gegaan bent.
%% 
%% Voorbeelden van zulke fasen zijn: literatuurstudie, opstellen van een
%% requirements-analyse, opstellen long-list (bij vergelijkende studie),
%% selectie van geschikte tools (bij vergelijkende studie, "short-list"),
%% opzetten testopstelling/PoC, uitvoeren testen en verzamelen
%% van resultaten, analyse van resultaten, ...
%%
%% !!!!! LET OP !!!!!
%%
%% Het is uitdrukkelijk NIET de bedoeling dat je het grootste deel van de corpus
%% van je bachelorproef in dit hoofstuk verwerkt! Dit hoofdstuk is eerder een
%% kort overzicht van je plan van aanpak.
%%
%% Maak voor elke fase (behalve het literatuuronderzoek) een NIEUW HOOFDSTUK aan
%% en geef het een gepaste titel.

\section{Functional requirements}
Voordat we een antwoord op de onderzoeksvraag kunnen vinden, is het noodzakelijk om eerst de verreisten duidelijk op te sommen. Deze vereisten zullen worden opgesteld volgens het MoSCoW-principe. De "Must-have"-vereisten zijn cruciaal en moeten worden vervuld door de engine. De "Should-have"-vereisten zijn belangrijk en kunnen doorslaggevend zijn. De "Could-have"-vereisten zijn optioneel maar zouden een toegevoegde waarde kunnen bieden. Er zullen geen `Wont-have` opgelijst worden aangezien AllPhi bij elke extra functionaliteit baat heeft.
\subsection{Must have}
\begin{itemize}
    \item De game engine moet de programmeertaal C\# ondersteunen. (M1)
    \item De game engine moet het ontwikkelen van 2D-games ondersteunen.(M2)
    \item De game engine moet het exporteren naar windows ondersteunen.(M3)
    \item De game engine moet een uitgebreide grafische ondersteuning aanbieden.(M4)
    \item De game engine moet een uitgebreide documentatie hebben.(M5)
    \item De game engine wordt nog actief ondersteund. (M6)
    \item De game engine is gratis. (M7)
\end{itemize}

\subsection{Should have}
\begin{itemize}
    \item De game engine heeft een brede 3de partij informatie.(S1)
    \item De game engine ondersteund het onwikkelen van 3D-games.(S2)
    \item De game engine is open source.(S3)
\end{itemize}

\subsection{Could Have}
\begin{itemize}
    \item De game engine ondersteund VR/AR voor latere projecten.(C1)
    \item De game engine ondersteund multiplayer functionaliteit.(C2)
\end{itemize}

\section{Het bepalen van geschikte game engines}
In dit onderdeel wordt de eerder gemaakte lijst van game engines vergeleken met de vereisten van AllPhi. De vereisten worden in de tabel met identificatoren weergegeven. De engines zullen normaal worden opgesomd. Hierdoor zal na vergelijking een kortere lijst resulteren. Met een van deze bekomen engines zal er een proof of concept uitgewerkt worden.
\\
\begin{center}
\begin{tabular}{ | m{8em} | m{1cm}| m{1cm} |m{1cm}| m{1cm}| m{1cm} |m{1cm}| m{1cm} |m{1cm}| m{1cm} | m{1cm} | } 
    \hline
     & A & B & C & D & E & F & G & H & I\\
    \hline
    C\# & X & X &  & X &  &  &  &  &  \\ 
    \hline
    2D & X & X & X & X & X & X & X &  & X \\ 
    \hline
    Windows & X & X & X & X & X & X & X & X & X \\ 
    \hline
    Gui & X & X & X & X & X & X & X & X & X \\ 
    \hline
    Documentatie & X & X & X & X & X & X & X & X & X \\ 
    \hline
    Actieve ondersteuning & X & X & X & X & X & X & X & X & X \\ 
    \hline
    gratis & X & X & X & X &  &  &  & X & X \\ 
    \hline
    3de partij informatie & X & X & X & X & X &  &  &  & X \\ 
    \hline
    3D & X & X &  &  &  &  &  & X & X \\ 
    \hline
    open source &  & X &  & X &  &  &  &  & X \\ 
    \hline
    VR/AR & X & X &  &  &  &  &  &  &  \\ 
    \hline
    Multiplayer & X & X & X & X & X & X & X & X & X \\ 
    \hline
\end{tabular}
\end{center}
\textbf{A} Unity,
\textbf{B} Godot
\textbf{C} Game Maker studio 2,
\textbf{D} Cocos2d-x,
\textbf{E} Construct 3,
\textbf{F} Phaser,
\textbf{G} GameSalad,
\textbf{H} Buildbox 3,
\textbf{I} GDevelop,
