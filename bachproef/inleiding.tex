%%=============================================================================
%% Inleiding
%%=============================================================================

\chapter{\IfLanguageName{dutch}{Inleiding}{Introduction}}%
\label{ch:inleiding}



\section{\IfLanguageName{dutch}{Probleemstelling}{Problem Statement}}%
\label{sec:probleemstelling}

AllPhi is een technologiebedrijf dat jaarlijks deelneemt aan verschillende technologiebeurzen, waar het een interactieve uitdaging voor bezoekers organiseert. Voor de meest recente editie van de uitdaging ontwikkelde AllPhi een eigen Snake-spel, gebouwd met behulp van C\# en .NET. Ondanks dat de medewerkers van AllPhi geen professionele gameontwikkelaars zijn, waren ze tevreden met het resultaat.

Het onderzoek zal resulteren in een beknopte lijst van geselecteerde game-engines die als potentieel geschikt worden beschouwd voor het ontwikkelen van een `Space Invaders`-achtig spel. Uit deze lijst zal een game-engine worden gekozen voor de ontwikkeling van een proof-of-concept. Dit proof-of-concept zal fungeren als praktisch demonstratiemiddel, waardoor AllPhi in staat zal zijn om nieuwe spellen te creëren voor gebruik op verschillende evenementen.


\section{\IfLanguageName{dutch}{Onderzoeksvraag}{Research question}}%
\label{sec:onderzoeksvraag}

Met als doel het verbeteren van toekomstige uitdagingen, wil AllPhi onderzoeken welke game-engine het meest geschikt is voor een arcade spel te maken die ze op verschillende evenementen kan gebruiken.

\section{\IfLanguageName{dutch}{Onderzoeksdoelstelling}{Research objective}}%
\label{sec:onderzoeksdoelstelling}

De kern van deze paper is het verstrekken van aanbevelingen aan AllPhi met betrekking tot geschikte game-engines voor de ontwikkeling van een `Space Invaders`-achtig spel. Dit wordt bereikt door middel van een grondig onderzoek dat een uitgebreide analyse omvat van beschikbare game-engines die geschikt zijn voor dit specifieke genre.

\section{\IfLanguageName{dutch}{Opzet van deze bachelorproef}{Structure of this bachelor thesis}}%
\label{sec:opzet-bachelorproef}

% Het is gebruikelijk aan het einde van de inleiding een overzicht te
% geven van de opbouw van de rest van de tekst. Deze sectie bevat al een aanzet
% die je kan aanvullen/aanpassen in functie van je eigen tekst.

De rest van deze bachelorproef is als volgt opgebouwd:

In Hoofdstuk~\ref{ch:stand-van-zaken} wordt een overzicht gegeven van de stand van zaken binnen het onderzoeksdomein, op basis van een literatuurstudie.

In Hoofdstuk~\ref{ch:methodologie} wordt de methodologie toegelicht en worden de gebruikte onderzoekstechnieken besproken om een antwoord te kunnen formuleren op de onderzoeksvragen.

In Hoofdstuk~\ref{ch:selectie-game-engines} wordt er een oplijsting gedaan van alle mogenlijke game engines die in aanmerking komen.



% TODO: Vul hier aan voor je eigen hoofstukken, één of twee zinnen per hoofdstuk
In Hoofdstuk~\ref{ch:proof-of-concept} wordt er een proof of concept uitgewerkt in een van de engines die geschikt zijn voor AllPhi.

In Hoofdstuk~\ref{ch:conclusie}, tenslotte, wordt de conclusie gegeven en een antwoord geformuleerd op de onderzoeksvragen. Daarbij wordt ook een aanzet gegeven voor toekomstig onderzoek binnen dit domein.