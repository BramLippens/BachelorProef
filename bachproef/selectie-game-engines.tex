%%=============================================================================
%% Methodologie
%%=============================================================================

\chapter{\IfLanguageName{dutch}{Selectie van game-engines}{Selection of Game-engines}}%
\label{ch:selectie-game-engines}

Na het vaststellen van de eisen waaraan de game-engine voor AllPhi moet voldoen, kunnen we beginnen met het selecteren van geschikte opties.In dit hoofdstuk zullen we eerst een uitgebreide lijst opstellen van alle engines die voldoen aan de gestelde criteria. Vervolgens wordt deze lijst verfijnd tot een shortlist met engines die voldoen aan alle essentiële eisen en zoveel mogelijk optionele wensen.

\section{Long-list}
In deze fase stellen we een zo uitgebreid mogelijke lijst op van game-engines die voldoen aan de eisen van AllPhi. We streven ernaar om alle beschikbare opties te omvatten, ongeacht of ze perfect aansluiten bij alle wensen.

\begin{itemize}
    \item Unity
    \item Godot
    \item Game Maker studio 2
    \item Cocos2d-x
    \item Construct 3
    \item Phaser
    \item GameSalad
    \item Buildbox
    \item GDevelop
    \item Infinity Engine
    \item Genie Engine
\end{itemize}

\section{Short-list}
Na het opstellen van de uitgebreide lijst, zullen we deze verfijnen tot een shortlist van engines die het meest geschikt zijn voor AllPhi.
\begin{itemize}
    \item Godot
    \item Unity
\end{itemize}