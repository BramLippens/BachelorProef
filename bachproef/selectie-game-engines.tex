%%=============================================================================
%% Methodologie
%%=============================================================================

\chapter{\IfLanguageName{dutch}{Selectie van game-engines}{Selection of Game-engines}}%
\label{ch:selectie-game-engines}

Na het vaststellen van de eisen waaraan de game-engine voor AllPhi dient te voldoen, kan worden overgegaan tot de selectie van geschikte opties. In dit hoofdstuk wordt eerst een uitgebreide lijst opgesteld van alle engines die aan de gestelde criteria voldoen. Vervolgens wordt deze lijst verfijnd tot een shortlist met engines die aan alle essentiële eisen en zoveel mogelijk optionele wensen tegemoetkomen.

\section{Long-list}

In deze fase wordt een zo uitgebreid mogelijke lijst opgesteld van game-engines die voldoen aan de eisen van AllPhi. Er wordt gestreefd naar het omvatten van alle beschikbare opties, ongeacht of deze perfect aansluiten bij alle wensen. Volgens J. Sibony zijn de volgende game engines een goed startpunt \autocite{Sibony2024}

\begin{itemize}
    \item Unity
    \item Godot
    \item Game Maker studio 2
    \item Cocos2d-x
    \item Construct 3
    \item Phaser
    \item GameSalad
    \item Buildbox
    \item GDevelop
    \item Infinity Engine
    \item Genie Engine
\end{itemize}

\section{Short-list}
Na het opstellen van de uitgebreide lijst, zullen we deze verfijnen tot een shortlist van engines die het meest geschikt zijn voor AllPhi.
\begin{itemize}
    \item Godot
    \item Unity
    \item Game Maker studio 2
\end{itemize}