\chapter{\IfLanguageName{dutch}{Stand van zaken}{State of the art}}%
\label{ch:stand-van-zaken}

% Tip: Begin elk hoofdstuk met een paragraaf inleiding die beschrijft hoe
% dit hoofdstuk past binnen het geheel van de bachelorproef. Geef in het
% bijzonder aan wat de link is met het vorige en volgende hoofdstuk.

% Pas na deze inleidende paragraaf komt de eerste sectiehoofding.
\section{Low Code}


\section{\IfLanguageName{dutch}{Game Engines}{Game Engines}}%
\label{sec:game-engines}
De term game engine is opgekomen rond midden 1990. Toen kwam het razend populaire videospel Doom gemaakt door ”id Software”. Doom was een van de spellen die de videospel industrie revolutioneert heeft. Het spel was gesplitst tussen 2 lagen. Hun kern softwarecomponenten zoals het renderen systeem van de drie dimensionale omgeving, de collisiedetectie en het geluidsysteem. De andere laag het design van het spel, de werelden/levels en de spelregels. Hierdoor kon ”id Software” nieuw iteraties van videospelen uitbrengen die er anders uitzagen, nieuwe unieke werelden en nieuwe spel regels zonder al te veel aanpassingen aan de kern componenten aan te brengen. Hieruit is dan term game engine ontstaan. \cite{gregory2018game}

\subsection{GameMaker studio 2}
GameMaker Studio 2, een engine die geschikt is voor beginners om 2D-games te maken, maakt gebruik van een drag-and-drop interface en vereist geen extra programmeertaal naast C\#. Het gebruikt zijn eigen taal genaamd GameMaker Language (GML) \autocite{cossu2019game}.

\subsection{Unity}
Unity, een commerciële game engine, biedt een breed scala aan functies, waaronder ondersteuning voor zowel 2D- als 3D-graphics, fysica, geluid, netwerken en augmented reality \autocite{Haas2014}.
\subsubsection{Unity blueprints}

\subsection{Godot}
Godot, een gratis en open-source platform, biedt een combinatie van toegankelijkheid en flexibiliteit, zonder verborgen functionaliteiten achter een betaalmuur. Het heeft ook een actieve en ondersteunende gemeenschap \autocite{Bradfield2018}.

\subsection{Unreal Engine 4}
Unreal Engine 4 omvat verschillende componenten zoals sound engines, physics engines, graphics engines, gameplay frameworks en online functionaliteiten. De engine bevat ook de Unreal Editor, waarin spellen worden geprogrammeerd met modules zoals Material en Blueprint \autocite{lee2016learning}.

\subsection{Buildbox}
Buildbox is ontworpen voor het ontwikkelen van 2D-games en maakt het mogelijk om snel spellen te maken en te exporteren naar verschillende platformen, inclusief mobiele telefoons \cite{audronis2016buildbox}.

\subsection{Construct 3}
Construct 3 is een HTML5-gebaseerde 2D-game engine die visueel programmeren ondersteunt en spellen kan exporteren naar Windows, MacOS, Linux en mobiele telefoons \cite{enwiki:1200994136}.