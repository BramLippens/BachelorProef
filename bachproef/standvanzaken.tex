\chapter{\IfLanguageName{dutch}{Stand van zaken}{State of the art}}%
\label{ch:stand-van-zaken}

% Tip: Begin elk hoofdstuk met een paragraaf inleiding die beschrijft hoe
% dit hoofdstuk past binnen het geheel van de bachelorproef. Geef in het
% bijzonder aan wat de link is met het vorige en volgende hoofdstuk.
In de huidige wereld is het een constante uitdaging om de aandacht van mensen vast te houden. Korte aandachtsspannen dwingen ons om creatiever en unieker te werk te gaan. Dit geldt ook voor bedrijven die softwareontwikkelaars willen aantrekken. Gelukkig biedt de wereld van gaming een gouden kans.
\\
De programmeerwereld telt een groot aantal enthousiaste gamers. Deze ontwikkelaars zijn gemotiveerd, creatief en op zoek naar nieuwe uitdagingen. Echter, niet alle bedrijven die deze doelgroep willen bereiken, beschikken over de kennis en kunde om zelf te gamen of game-engines te gebruiken.
\\
Game-engines bieden een unieke oplossing. Deze krachtige tools stellen ontwikkelaars in staat om boeiende en interactieve applicaties te creëren zonder diepgaande gamedesignkennis te hoeven bezitten. Met behulp van diverse tools en features stroomlijnt de ontwikkeling van prototypes, simulaties en visuele interfaces.

\section{Theoretisch Kader}
TODO fotos toevoegen
\subsection{\IfLanguageName{dutch}{Game Engine}{Game Engine}}%
\label{sec:game-engines}
De term "game engine" kwam op in het midden van de jaren 1990, met de opkomst van het videospel Doom, ontwikkeld door id Software. Doom was een van de spellen die een revolutie teweegbrachten in de videogame-industrie. Het spel was opgebouwd uit twee lagen: de kernsoftwarecomponenten, zoals het renderingsysteem voor de driedimensionale omgeving, de collisiedetectie en het geluidssysteem, en de andere laag, die het ontwerp van het spel, de werelden/niveaus en de spelregels omvatte. Hierdoor kon id Software nieuwe iteraties van videospellen uitbrengen met verschillende looks, nieuwe unieke werelden en nieuwe spelregels, zonder ingrijpende aanpassingen aan de kerncomponenten te hoeven maken. Dit leidde tot de term "game engine". \cite{gregory2018game}

\subsection{Scenes}%
In de dynamische wereld van game development fungeren scenes als de fundamentele bouwstenen van een meeslepende spelomgeving. Net zoals hoofdstukken de structuur van een boek bepalen, vormen scenes de distinct afgebakende elementen die samen de unieke game-ervaring creëren. Van interactieve menu's tot uitdagende levels, cutscenes die het verhaal vertellen en boeiende interactieve omgevingen, elke scene draagt bij aan de cohesie en immersie van de game.

Scenes vervullen diverse cruciale functies die de ontwikkeling en het uiteindelijke spel optimaliseren. Ze brengen orde in de complexiteit dat soms spellen hebben. Door een spel op te splitsen in verschillende scenes behouden de ontwikkelaars een helder overzicht en kunnen ze efficiënter werken. Dit vereenvoudigt het beheer van de verschillende elementen.

\subsection{Sprite}
In de dynamische wereld van computer graphics zijn sprites onmisbare elementen die leven en beweging brengen in digitale ruimtes. Deze tweedimensionale bitmapafbeeldingen, vaak met transparantie, vormen de basis voor talloze visuele effecten, van bewegende karakters in videogames tot iconen in gebruikersinterfaces.

De kracht van sprites ligt in hun eenvoud en efficiëntie. Door snel opeenvolgende beelden te tonen, creëren ze de illusie van beweging en voegen ze dynamiek toe aan statische beelden. Dit maakt ze ideaal voor het weergeven van complexe elementen zonder de prestaties van computers te belasten.

\subsection{\IfLanguageName{dutch}{Collisie Detectie}{Collision Detection}}
In de dynamische wereld van videogames is Collision Detection (CD) een onmisbaar element dat realistische interacties tussen objecten mogelijk maakt. Het fungeert als de onzichtbare hand die ervoor zorgt dat spelers, vijanden, projectielen en de omgeving op een natuurlijke en geloofwaardige manier met elkaar omgaan, waardoor meeslepende en interactieve spelervaringen gecreëerd worden.
CD detecteert wanneer twee of meer objecten in een game met elkaar in botsing komen. Dit omvat zowel statische objecten (zoals muren, vloeren en obstakels) die de grenzen van de spelwereld definiëren, als dynamische objecten (zoals de speler, vijanden en projectielen) die voortdurend in beweging zijn.
Door deze botsingen te detecteren, kan de game op diverse manieren reageren op deze interacties, wat realistische consequenties en spannende gameplay-momenten creëert.

% Pas na deze inleidende paragraaf komt de eerste sectiehoofding.
\subsection{Low Code}
Low-code is softwarebenadering dat bijna tot geen programmeer ervaring nodig heeft om iets te ontwikkelen. In plaats van complexe programmeer talen te leren en gebruiken, is er een visuele interface.\autocite{Kissflow2024} Met lowcode sleep je eenvoudig elementen in je werkveld en koppel je ze met elkaar. Dit maakt het ontwikkelingsproces intuïtief en meer toegankelijk. Als je op deze manier een spel zou ontwikkelen moet je dus niet de achterliggende code van de game engine leren maar alleen de interface. Hierdoor ligt de focus meer bij het ontwerpen en hoe de objecten met elkaar werken. Nog een pluspunt is dat bij gebruik van low code game engines is dat de duur om een spel te ontwikkelen lager ligt dan op de traditionele manier. \autocite{Trigo2022}

\section{Unity en CryEngine}
Allphi heeft als een van de vereisten van een game engine dat deze in de programmeer taal C\# ondersteund. Twee game engines die al in aanmerking komen zijn Unity en CryEngine. Unity is gemaakt door Unity Technologies. Met deze engine kan je zowel 3D als 2D games maken. Dit is een pluspunt voor AllPhi want zij willen namelijk starten met 2D games maar de mogenlijkheid hebben om in de toekomst 3D games te maken. Bij de opstart van Unity ondersteunde ze alleen het platform Mac OS. In verdere uitgaves hebben ze hun ondersteuning uitgebreid en ondersteunen ze nu een breed gala aan platformen waaronder Windows. CryEngine ondersteund C\#. Deze engine is ontwikkeld door Crytek. Hun origineel plan was om alleen een demo te maken. Ze hebben dit verder uitgebreid naar een volledige engine. In het begin was het alleen mogenlijk om First-person shooter spellen te ontwikkelen, maar in 2016 hebben ze het mogenlijk gemaakt om verschillende genres te maken. CryEngine ondersteund verschillende platformen waaronder Windows dat Allphi wenst. Jammer genoeg kan je alleen maar 3D games ontwikkelen binnen CryEngine. \autocite{Barczak2021}

\section{Vergelijkende studie tussen Unity en Godot}
Godot blinkt uit in gebruiksvriendelijkheid. De IDE is intuïtiever en eenvoudiger te begrijpen, mede dankzij de ingebouwde script-editor. Dit maakt het voor beginners een stuk laagdrempeliger om aan de slag te gaan. Unity heeft daarentegen de overhand op het gebied van documentatie. Hun handleidingen zijn uitgebreid en cateren aan verschillende niveaus van kennis. \autocite{flomen2020game}

\section{Beschikbare game engines}
\subsection{GameMaker studio 2}
GameMaker Studio 2, een engine die geschikt is voor beginners om 2D-games te maken, maakt gebruik van een drag-and-drop interface ook wel ``visual scripting language''  genoemd. Hierdoor kun je videospellen ontwikkelen zonder het schrijven van code. Je codeert door niet in taal maar in visuele aspecten. en vereist geen extra programmeertaal naast C\#. Hierdoor is het een engine die in aanmerking komt voor AllPhi. Zij zijn beginners in de game ontwikkeling wereld. \autocite{cossu2019game}

\subsection{Godot}
Godot, een gratis en open-source platform, biedt een combinatie van toegankelijkheid en flexibiliteit, zonder verborgen functionaliteiten achter een betaalmuur. Wat Godot verder onderscheidt, is de levendige en ondersteunende gemeenschap die het omringt. Deze gemeenschap draagt bij aan het continue verbeteren en uitbreiden van de mogelijkheden van het platform, waardoor ontwikkelaars profiteren van een schat aan kennis, ondersteuning en gedeelde middelen. \autocite{Bradfield2018}

In Godot vormen nodes de bouwstenen voor het maken van spellen. Een node is een object dat gespecialiseerde spelfunctionaliteit vertegenwoordigt, zoals het weergeven van een scherm, afspelen van animaties of modellen. Elke node heeft verschillende eigenschappen waarmee het gedrag kan worden geconfigureerd. Het modulaire systeem maakt het toevoegen van nodes flexibel en efficiënt. \autocite{Bradfield2018}

Het scripten in Godot kan worden gedaan in drie verschillende talen. GDScript is de standaardtaal binnen Godot en staat het dichtst bij de engine zelf. C# is een alternatief van GDScript. Dit is voor AllPhi een voordeel omdat zij met deze taal gekend zijn. Godot ondersteunt echter een flexibele combinatie van deze twee talen.\autocite{Bradfield2018} VisualScript was een derde optie, maar deze taal wordt niet langer ondersteund sinds de release van Godot 4.0.

\subsection{Unreal Engine 4}
Unreal Engine 4 omvat een breed scala aan componenten die verschillende aspecten van game-ontwikkeling bestrijken, waaronder sound engines, physics engines, graphics engines, gameplay frameworks en online functionaliteiten. Een van de meest opvallende kenmerken is de Unreal Editor, een krachtige ontwikkelomgeving waarin spellen worden geprogrammeerd en ontworpen. De editor biedt een reeks modules, zoals Material en Blueprint, die ontwikkelaars in staat stellen om op een intuïtieve manier complexe game-elementen te maken en te beheren. Met deze uitgebreide set tools en functies stelt Unreal Engine 4 ontwikkelaars in staat om hoogwaardige, interactieve ervaringen te creëren binnen een geavanceerde en veelzijdige ontwikkelomgeving. \autocite{lee2016learning}

\subsubsection{Blueprints}
Het woord "Blueprint" heeft verschillende betekenissen binnen Unreal Engine 4. Aan de ene kant verwijst het naar een door Epic Games ontwikkelde scripttaal specifiek voor Unreal Engine 4. Aan de andere kant wordt het gebruikt om te verwijzen naar een nieuw type spelobject dat is gemaakt met behulp van Blueprint-functionaliteit.
\autocite{romero2019blueprints}

\subsection{Buildbox}
Buildbox is specifiek ontworpen voor het ontwikkelen van 2D-games en staat bekend om zijn vermogen om ontwikkelaars in staat te stellen snel spellen te creëren en te exporteren naar diverse platforms, waaronder mobiele telefoons. Met een intuïtieve interface en een krachtige set aan tools, stelt Buildbox gebruikers in staat om zonder uitgebreide programmeerkennis complexe 2D-games te ontwikkelen. Het platform biedt ook uitgebreide mogelijkheden voor het aanpassen van gameplay-elementen en het ontwerpen van visueel aantrekkelijke spelomgevingen, waardoor ontwikkelaars hun creativiteit volledig kunnen benutten. Met Buildbox kunnen ontwikkelaars snel en efficiënt hoogwaardige 2D-games produceren die geschikt zijn voor een breed scala aan apparaten en platforms. \autocite{audronis2016buildbox}

\subsection{Construct 3}
Construct 3 is een geavanceerde HTML5-gebaseerde 2D-game engine die bekend staat om zijn ondersteuning voor visueel programmeren. Met een intuïtieve interface en krachtige tools stelt Construct 3 gebruikers in staat om spellen te ontwikkelen zonder dat er diepgaande programmeerkennis vereist is. Een opvallend kenmerk van Construct 3 is zijn vermogen om spellen te exporteren naar een breed scala aan platforms, waaronder Windows, MacOS, Linux en mobiele telefoons. Deze flexibiliteit stelt ontwikkelaars in staat om hun creaties gemakkelijk beschikbaar te maken voor een groot publiek, ongeacht het apparaat dat ze gebruiken. Met Construct 3 kunnen ontwikkelaars snel en efficiënt hoogwaardige 2D-games maken en deze distribueren naar verschillende doelplatforms. \autocite{enwiki:1200994136}

\subsection{GDevelop}
GDevelop is ontwikkeld door Florian Rival, een ontwikkelaar bij Google. Het is een grafische en open-source game engine. Het heeft een grafische gebruikers interface. Hierdoor moet men niet de achterliggende taal kennen om met deze engine een spel te ontwikkelen. Het gebruikt event-sheets, een lijst van allemaal evenementen. Hierdoor kan je spellen ontwikkelen door middel van drag-and-drop principe. Evenementen zijn opgesteld uit verschillende condities en acties. Een nadeel van deze engine is dat de objecten heel generiek zijn. Hierdoor kan de vrijheid gelimiteerd zijn. Maar voor simpele spellen te ontwikkelen moet dit geen probleem vormen.\autocite{mohd2023analyzing}