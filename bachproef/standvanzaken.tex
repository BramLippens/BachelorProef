\chapter{\IfLanguageName{dutch}{Stand van zaken}{State of the art}}%
\label{ch:stand-van-zaken}

% Tip: Begin elk hoofdstuk met een paragraaf inleiding die beschrijft hoe
% dit hoofdstuk past binnen het geheel van de bachelorproef. Geef in het
% bijzonder aan wat de link is met het vorige en volgende hoofdstuk.

% Pas na deze inleidende paragraaf komt de eerste sectiehoofding.
\section{Low Code}
Low-code is softwarebenadering dat bijna tot geen programmeer ervaring nodig heeft om iets te ontwikkelen. In plaats van complexe programmeer talen te leren en gebruiken, is er een visuele interface.\autocite{Kissflow2024} Met lowcode sleep je eenvoudig elementen in je werkveld en koppel je ze met elkaar. Dit maakt het ontwikkelingsproces intuïtief en meer toegankelijk. Als je op deze manier een spel zou ontwikkelen moet je dus niet de achterliggende code van de game engine leren maar alleen de interface. Hierdoor ligt je focus meer bij het ontwerpen en hoe de objecten met elkaar werken. Nog een pluspunt is dat bij gebruik van low code game engines is dat de duur om een spel te ontwikkelen lager ligt dan op de traditionele manier.

\section{\IfLanguageName{dutch}{Game Engines}{Game Engines}}%
\label{sec:game-engines}
De term "game engine" kwam op in het midden van de jaren 1990, met de opkomst van het razend populaire videospel Doom, ontwikkeld door id Software. Doom was een van de spellen die een revolutie teweegbrachten in de videogame-industrie. Het spel was opgebouwd uit twee lagen: de kernsoftwarecomponenten, zoals het renderingsysteem voor de driedimensionale omgeving, de collisiedetectie en het geluidssysteem, en de andere laag, die het ontwerp van het spel, de werelden/niveaus en de spelregels omvatte. Hierdoor kon id Software nieuwe iteraties van videospellen uitbrengen met verschillende looks, nieuwe unieke werelden en nieuwe spelregels, zonder ingrijpende aanpassingen aan de kerncomponenten te hoeven maken. Dit leidde tot de term "game engine". \cite{gregory2018game}

\subsection{GameMaker studio 2}
GameMaker Studio 2, een engine die geschikt is voor beginners om 2D-games te maken, maakt gebruik van een drag-and-drop interface en vereist geen extra programmeertaal naast C\#. Het gebruikt zijn eigen taal genaamd GameMaker Language (GML). \autocite{cossu2019game}

\subsection{Unity}
Een ander opmerkelijk voorbeeld is Unity, een commerciële game engine die een breed scala aan functionaliteiten biedt. Naast ondersteuning voor zowel 2D- als 3D-graphics, biedt Unity ook geavanceerde mogelijkheden op het gebied van fysica, geluid, netwerken en augmented reality. Dit maakt Unity een populaire keuze onder ontwikkelaars voor het creëren van diverse en innovatieve game-ervaringen. \autocite{Haas2014}

\subsection{Godot}
Godot, een gratis en open-source platform, biedt een uitstekende combinatie van toegankelijkheid en flexibiliteit, zonder verborgen functionaliteiten achter een betaalmuur. Wat Godot verder onderscheidt, is de levendige en ondersteunende gemeenschap die het omringt. Deze gemeenschap draagt bij aan het continue verbeteren en uitbreiden van de mogelijkheden van het platform, waardoor ontwikkelaars profiteren van een schat aan kennis, ondersteuning en gedeelde middelen. \autocite{Bradfield2018}

In Godot vormen nodes de bouwstenen voor het maken van spellen. Een node is een object dat gespecialiseerde spelfunctionaliteit vertegenwoordigt, zoals het weergeven van een scherm, afspelen van animaties of modellen. Elke node heeft verschillende eigenschappen waarmee het gedrag kan worden geconfigureerd. Het modulaire systeem maakt het toevoegen van nodes flexibel en efficiënt. \autocite{Bradfield2018}

Het scripten in Godot kan worden gedaan in drie verschillende talen. GDScript is de standaardtaal binnen Godot en staat het dichtst bij de engine zelf. C# kan ook worden gebruikt, maar wordt meestal ingezet voor optimalisatiedoeleinden. Godot ondersteunt echter een flexibele combinatie van deze twee talen.\autocite{Bradfield2018} VisualScript was een derde optie, maar deze taal wordt niet langer ondersteund sinds de release van Godot 4.0.

\subsection{Unreal Engine 4}
Unreal Engine 4 omvat een breed scala aan componenten die verschillende aspecten van game-ontwikkeling bestrijken, waaronder sound engines, physics engines, graphics engines, gameplay frameworks en online functionaliteiten. Een van de meest opvallende kenmerken is de Unreal Editor, een krachtige ontwikkelomgeving waarin spellen worden geprogrammeerd en ontworpen. De editor biedt een reeks modules, zoals Material en Blueprint, die ontwikkelaars in staat stellen om op een intuïtieve manier complexe game-elementen te maken en te beheren. Met deze uitgebreide set tools en functies stelt Unreal Engine 4 ontwikkelaars in staat om hoogwaardige, interactieve ervaringen te creëren binnen een geavanceerde en veelzijdige ontwikkelomgeving. \autocite{lee2016learning}

\subsubsection{Blueprints}
Het woord "Blueprint" heeft verschillende betekenissen binnen Unreal Engine 4. Aan de ene kant verwijst het naar een door Epic Games ontwikkelde scripttaal specifiek voor Unreal Engine 4. Aan de andere kant wordt het gebruikt om te verwijzen naar een nieuw type spelobject dat is gemaakt met behulp van Blueprint-functionaliteit.
\autocite{romero2019blueprints}

\subsection{Buildbox}
Buildbox is specifiek ontworpen voor het ontwikkelen van 2D-games en staat bekend om zijn vermogen om ontwikkelaars in staat te stellen snel spellen te creëren en te exporteren naar diverse platforms, waaronder mobiele telefoons. Met een intuïtieve interface en een krachtige set aan tools, stelt Buildbox gebruikers in staat om zonder uitgebreide programmeerkennis complexe 2D-games te ontwikkelen. Het platform biedt ook uitgebreide mogelijkheden voor het aanpassen van gameplay-elementen en het ontwerpen van visueel aantrekkelijke spelomgevingen, waardoor ontwikkelaars hun creativiteit volledig kunnen benutten. Met Buildbox kunnen ontwikkelaars snel en efficiënt hoogwaardige 2D-games produceren die geschikt zijn voor een breed scala aan apparaten en platforms. \autocite{audronis2016buildbox}

\subsection{Construct 3}
Construct 3 is een geavanceerde HTML5-gebaseerde 2D-game engine die bekend staat om zijn ondersteuning voor visueel programmeren. Met een intuïtieve interface en krachtige tools stelt Construct 3 gebruikers in staat om spellen te ontwikkelen zonder dat er diepgaande programmeerkennis vereist is. Een opvallend kenmerk van Construct 3 is zijn vermogen om spellen te exporteren naar een breed scala aan platforms, waaronder Windows, MacOS, Linux en mobiele telefoons. Deze flexibiliteit stelt ontwikkelaars in staat om hun creaties gemakkelijk beschikbaar te maken voor een groot publiek, ongeacht het apparaat dat ze gebruiken. Met Construct 3 kunnen ontwikkelaars snel en efficiënt hoogwaardige 2D-games maken en deze distribueren naar verschillende doelplatforms. \autocite{enwiki:1200994136}

\subsection{GDevelop}
GDevelop is ontwikkeld door Florian Rival, een ontwikkelaar bij Google. Het is een grafische en open-source game engine. Het heeft een grafische gebruikers interface. Hierdoor moet men niet de achterliggende taal kennen om met deze engine een spel te ontwikkelen. Het gebruikt event-sheets, een lijst van allemaal evenementen. Hierdoor kan je spellen ontwikkelen door middel van drag-and-drop principe. Evenementen zijn opgesteld uit verschillende condities en acties. Een nadeel van deze engine is dat de objecten heel generiek zijn. Hierdoor kan de vrijheid gelimiteerd zijn. Maar voor simpele spellen te ontwikkelen moet dit geen probleem vormen.\autocite{mohd2023analyzing}