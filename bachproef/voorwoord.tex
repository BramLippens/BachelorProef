%%=============================================================================
%% Voorwoord
%%=============================================================================

\chapter*{\IfLanguageName{dutch}{Woord vooraf}{Preface}}%
\label{ch:voorwoord}

%% TODO:
%% Het voorwoord is het enige deel van de bachelorproef waar je vanuit je
%% eigen standpunt (``ik-vorm'') mag schrijven. Je kan hier bv. motiveren
%% waarom jij het onderwerp wil bespreken.
%% Vergeet ook niet te bedanken wie je geholpen/gesteund/... heeft

Als gedreven student Toegepaste Informatica aan de Hogeschool Gent (HoGent) ben ik altijd op zoek naar uitdagingen die me intellectueel stimuleren en mijn kennis in de praktijk brengen. De vraag "Welke 2D game engines zijn geschikt voor Allphi-programmeurs met een C#-achtergrond, maar zonder game development ervaring, om games te ontwikkelen voor evenementen?" trok me dan ook meteen aan.

Deze keuze was niet enkel ingegeven door mijn fascinatie voor nieuwe technologieën, maar ook door mijn langdurige interesse in game development. De unieke kans om mijn C#-kennis te combineren met mijn passie voor games, maakte dit project voor mij uiterst waardevol en persoonlijk relevant.

De totstandkoming van deze bachelorproef was enkel mogelijk dankzij de steun en begeleiding van diverse personen. Graag wil ik dan ook mijn oprechte dankbaarheid uitspreken aan mijn promotor en co-promotor voor hun waardevolle feedback en continue ondersteuning tijdens het gehele proces. Hun expertise heeft een significante bijdrage geleverd aan de kwaliteit van deze proef. Mijn familie voor hun onvoorwaardelijke steun en moediging gedurende de uitdagende momenten. Jullie geloof in mij heeft me gemotiveerd om te volharden en het beste van mezelf te geven. De werknemers binnen AllPhi voor hun nuttige inzichten en raadgevingen wanneer ik vastliep. Jullie collegialiteit en behulpzaamheid waren hartverwarmend.

Het maken van deze bachelorproef was een leerrijke en uitdagende reis. De inzichten die ik heb opgedaan en de vaardigheden die ik heb ontwikkeld, zullen ongetwijfeld van grote waarde zijn in mijn toekomstige carrière als softwareontwikkelaar.