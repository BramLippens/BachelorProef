%---------- Inleiding ---------------------------------------------------------

\section{Introductie}%
\label{sec:introductie}

AllPhi is een technologiebedrijf dat jaarlijks deelneemt aan verschillende technologiebeurzen, waar het een interactieve uitdaging voor bezoekers organiseert. Voor de meest recente editie van de uitdaging ontwikkelde AllPhi een eigen Snake-spel, gebouwd met behulp van C\# en .NET. Ondanks dat de medewerkers van AllPhi geen professionele gameontwikkelaars zijn, waren ze tevreden met het resultaat.

Met als doel het verbeteren van toekomstige uitdagingen, wil AllPhi onderzoeken welke game-engine het meest toegankelijk is voor mensen met een achtergrond in C\# en .NET, maar zonder ervaring in game-ontwikkeling.

De kern van deze bachelorsproef is het verstrekken van aanbevelingen aan AllPhi met betrekking tot geschikte game-engines voor de ontwikkeling van een `Space Invaders`-achtig spel. Dit wordt bereikt door middel van een grondig onderzoek dat een uitgebreide analyse omvat van beschikbare game-engines die geschikt zijn voor dit specifieke genre.

Het onderzoek zal resulteren in een beknopte lijst van geselecteerde game-engines die als potentieel geschikt worden beschouwd voor het ontwikkelen van een `Space Invaders`-achtig spel. Uit deze lijst zal een game-engine worden gekozen voor de ontwikkeling van een proof-of-concept. Dit proof-of-concept zal fungeren als praktisch demonstratiemiddel, waardoor AllPhi in staat zal zijn om nieuwe spellen te creëren voor gebruik op verschillende evenementen.


%---------- Stand van zaken ---------------------------------------------------

\section{State-of-the-art}%
\label{sec:state-of-the-art}

Het onderzoek omvat een analyse van verschillende game engines die mogelijk geschikt zijn voor het ontwikkelen van het beoogde `Space Invaders`-type spel. Een beknopt overzicht van de geselecteerde game engines omvat:

GameMaker Studio 2, een engine die geschikt is voor beginners om 2D-games te maken, maakt gebruik van een drag-and-drop interface en vereist geen extra programmeertaal naast C\#. Het gebruikt zijn eigen taal genaamd GameMaker Language (GML) \autocite{cossu2019game}.

Unity, een commerciële game engine, biedt een breed scala aan functies, waaronder ondersteuning voor zowel 2D- als 3D-graphics, fysica, geluid, netwerken en augmented reality \autocite{Haas2014}.

Godot, een gratis en open-source platform, biedt een combinatie van toegankelijkheid en flexibiliteit, zonder verborgen functionaliteiten achter een betaalmuur. Het heeft ook een actieve en ondersteunende gemeenschap \autocite{Bradfield2018}.

Unreal Engine 4 omvat verschillende componenten zoals sound engines, physics engines, graphics engines, gameplay frameworks en online functionaliteiten. De engine bevat ook de Unreal Editor, waarin spellen worden geprogrammeerd met modules zoals Material en Blueprint \autocite{lee2016learning}.

Buildbox is ontworpen voor het ontwikkelen van 2D-games en maakt het mogelijk om snel spellen te maken en te exporteren naar verschillende platformen, inclusief mobiele telefoons \cite{audronis2016buildbox}.

Construct 3 is een HTML5-gebaseerde 2D-game engine die visueel programmeren ondersteunt en spellen kan exporteren naar Windows, MacOS, Linux en mobiele telefoons \cite{enwiki:1200994136}.

Deze overzicht biedt een eerste inzicht in de diverse opties die beschikbaar zijn voor de ontwikkeling van het gewenste spel, en elk van deze engines zal nader worden geëvalueerd in overeenstemming met de specifieke criteria en behoeften van AllPhi.






% Voor literatuurverwijzingen zijn er twee belangrijke commando's:
% \autocite{KEY} => (Auteur, jaartal) Gebruik dit als de naam van de auteur
%   geen onderdeel is van de zin.
% \textcite{KEY} => Auteur (jaartal)  Gebruik dit als de auteursnaam wel een
%   functie heeft in de zin (bv. ``Uit onderzoek door Doll & Hill (1954) bleek
%   ...'')

%---------- Methodologie ------------------------------------------------------
\section{Methodologie}
\label{sec:methodologie}

Het onderzoek is gestructureerd in drie opeenvolgende fasen, die nauwgezet op elkaar zullen volgen om een grondige en systematische analyse van het spel en zijn vereisten te waarborgen.

In de eerste fase zal uitgebreide informatie worden verzameld om te bepalen welke game-engines voldoen aan een uitgebreide reeks essentiële criteria, zoals vastgesteld volgens het \\MoSCoW-principe. Deze criteria zullen in overleg met AllPhi worden vastgesteld, met als doel een grondig begrip te krijgen van hun specifieke vereisten en voorkeuren met betrekking tot het spel. Gedurende een geschatte periode van 2 tot 3 weken zal deze fase zich richten op het verkennen van verschillende opties.

De tweede fase van het onderzoek zal zich richten op het verfijnen van de initiële lijst van game-engines tot een meer beheersbare selectie, waarbij de nadruk ligt op het identificeren van die engines die voldoen aan een groot aantal should- en could-have criteria voor het spel. Deze fase, gepland over een periode van 2 weken, zal een diepere analyse omvatten van de potentieel geschikte opties die voortkomen uit de eerste fase, waarbij aandacht wordt besteed aan aspecten zoals functionaliteit, gebruiksgemak en technische ondersteuning.

Ten slotte zal de derde fase van het onderzoek zich concentreren op de praktische uitvoering door middel van het ontwikkelen van een proof-of-concept met behulp van een zorgvuldig geselecteerde game-engine uit de kortere lijst. Dit proces zal in nauwe samenwerking met AllPhi worden uitgevoerd om ervoor te zorgen dat het eindproduct volledig voldoet aan hun specifieke behoeften en verwachtingen met betrekking tot het spel. Gezien de complexiteit en de omvang van deze fase wordt een aanzienlijke tijdsperiode van 6 tot 8 weken gereserveerd om een grondige en zorgvuldige ontwikkeling te garanderen.


%---------- Verwachte resultaten ----------------------------------------------
\section{Verwacht resultaat, conclusie}%
\label{sec:verwachte_resultaten}

In de eerste fase zal een uitgebreide lijst van game-engines worden samengesteld, die voldoen aan de basisvereisten die AllPhi heeft gesteld. Deze lijst zal dienen als een brede selectie van potentiële opties voor toekomstige projecten, waarbij criteria zoals ondersteuning voor C\# en .NET, beschikbaarheid van documentatie en community-ondersteuning, en de mogelijkheid om 2D-games te ontwikkelen worden gehanteerd.

Vervolgens zal de tweede fase zich concentreren op het verfijnen van deze grote lijst tot een kortere selectie van game-engines die voldoen aan een uitgebreidere reeks criteria. Deze criteria zullen niet alleen de basisvereisten omvatten, maar ook meer geavanceerde functies en mogelijkheden die relevant zijn voor het specifieke spel dat AllPhi wil ontwikkelen. Deze verfijnde lijst zal AllPhi in staat stellen om het spel zelf te reproduceren, mochten ze dat willen.

Ten slotte zal de derde fase resulteren in de ontwikkeling van een proof-of-concept met behulp van de gekozen game-engine uit de kortere lijst. Dit proof-of-concept zal dienen als een praktisch demonstratiemiddel om de haalbaarheid en potentie van het spel te tonen, zowel voor interne als externe stakeholders. Het biedt AllPhi de mogelijkheid om het spel te presenteren en te evalueren op evenementen, waardoor ze een tastbare demonstratie hebben van hun game-ontwikkelingscapaciteiten.


