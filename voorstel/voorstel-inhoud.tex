%---------- Inleiding ---------------------------------------------------------

\section{Introductie}%
\label{sec:introductie}

AllPhi is een technologiebedrijf dat elk jaar aanwezig is op een aantal technologiebeurzen. Op deze beurzen organiseert AllPhi een interactieve challenge voor bezoekers. De winnaar van de challenge ontvangt een leuke prijs.

Voor de meest recente editie van de challenge ontwikkelde AllPhi een eigen Snake-spel. Dit spel was ontwikkeld met behulp van C\# en .NET. Hoewel AllPhi geen professionele gameontwikkelaars zijn, waren ze tevreden met het resultaat.

Om in de toekomst nog betere challenges te kunnen ontwikkelen, wil AllPhi onderzoeken welke game engine het makkelijkst te gebruiken is voor mensen met een achtergrond in C\# en .NET, maar geen game development achtergrond hebben.


%---------- Stand van zaken ---------------------------------------------------

\section{State-of-the-art}%
\label{sec:state-of-the-art}
\subsection{Game engines}
Game engines zijn een cruciaal element in de game development wereld. Deze vergemakkelijken het ontwikkelen van videospellen. Ze bieden varierende tools aan om dit te bereiken. Van animaties tot gebruikers interacties tot detectie van botsingen. Volgens \textcite{Barczak2021} is een game engine een software toolkit die productie van games op verschillende platformen vereenvoudigt.

\subsection{Unity}
Unity is een commerciële game engine die een breed scala aan features biedt, waaronder ondersteuning voor 2D- en 3D-graphics, fysica, geluid, netwerken en augmented reality.\autocite{Haas2014} Unity is echter ook duurder en heeft een hogere leercurve dan Godot.

\subsection{Godot}
Godot onderscheidt zich als een volledig uitgeruste, moderne game engine die een unieke combinatie van toegankelijkheid en flexibiliteit biedt. Als een gratis en open-source platform, biedt Godot een hoger niveau van transparantie in vergelijking met commerciële alternatieven zoals Unity. Deze openheid zorgt ervoor dat er geen functionaliteiten verborgen zijn achter een betaalmuur, wat een belangrijk voordeel is voor ontwikkelaars die met budgetbeperkingen werken.\autocite{Bradfield2018}

Godot's open-source karakter zorgt ook voor een actieve en ondersteunende gemeenschap. Deze gemeenschap speelt een cruciale rol bij het voortdurend verbeteren van de engine, het bijdragen aan documentatie, en het delen van kennis, wat van onschatbare waarde is voor zowel nieuwe als ervaren gebruikers.

% Voor literatuurverwijzingen zijn er twee belangrijke commando's:
% \autocite{KEY} => (Auteur, jaartal) Gebruik dit als de naam van de auteur
%   geen onderdeel is van de zin.
% \textcite{KEY} => Auteur (jaartal)  Gebruik dit als de auteursnaam wel een
%   functie heeft in de zin (bv. ``Uit onderzoek door Doll & Hill (1954) bleek
%   ...'')

%---------- Methodologie ------------------------------------------------------
\section{Methodologie}
\label{sec:methodologie}

Deze studie gebruikt een vergelijkende benadering om twee identieke 2D-platformspellen te ontwikkelen, elk met behulp van een andere game engine: Godot en Unity. Het primaire doel is het onderzoeken van een scala aan variabelen, waaronder gebruiksgemak, leercurve voor beginners, technologische integratie, tijd benodigd voor implementatie, en de totale gebruikerservaring. Hieronder volgt een gedetailleerd overzicht van de geplande functionaliteiten van de spellen en de evaluatiecriteria.

In de eerste fase worden beide game engines geïnstalleerd, waarvoor een periode van één week is gereserveerd. Voor de implementatie van elke functionaliteit wordt vervolgens 1 tot 2 weken uitgetrokken. Dit tijdsbestek zorgt ervoor dat er voldoende ruimte is om elke functie in beide engines te implementeren en de resultaten grondig te analyseren.

\subsection{Functionaliteiten voor de Proof-of-Concept Spellen}

De proof-of-concept spellen zullen worden voorzien van diverse essentiële kenmerken om een representatieve vergelijking tussen Godot en Unity mogelijk te maken. Deze kenmerken omvatten:

\begin{itemize}
    \item \textbf{Beweging:} Het karakter kan bewegen naar links en rechts met behulp van de pijltoetsen of WASD-toetsen.
    \item \textbf{Wereld:} Het eerste level vindt plaats in een 2D-wereld met blokken en andere objecten.
    \item \textbf{Springen en Duiken:} Het karakter kan acties uitvoeren zoals springen en duiken.
    \item \textbf{Interactie met Blokken:} Het karakter kan interageren met omgevingsblokken, bijvoorbeeld klimmen, verplaatsen of vernietigen.
    \item \textbf{Interactie met Vijanden:} Het karakter kan vijanden verslaan of ontwijken.
    \item \textbf{Scoresysteem:} Een systeem dat scores berekent op basis van verslagen vijanden en verzamelde blokken.
    \item \textbf{Levelselectie:} Een menu om verschillende speellevels te selecteren.
    \item \textbf{Moeilijkheidsgraden:} Verschillende moeilijkheidsniveaus aangepast aan de vaardigheden van de speler.
\end{itemize}

Elke functionaliteit wordt afwisselend eerst in één van de twee engines geïmplementeerd. Na implementatie in beide engines worden de volgende criteria geëvalueerd:
\begin{itemize}
    \item \textbf{Implementatietijd:} De tijd benodigd voor de implementatie van elke functionaliteit.
    \item \textbf{Leerdurve:} De moeilijkheidsgraad bij het implementeren van de functionaliteit.
    \item \textbf{Gebruikerstests:} Deze worden uitgevoerd om de gebruikerservaring van beide game engines te evalueren.
\end{itemize}

Deze methodologische aanpak biedt een gestructureerde en evenwichtige vergelijking van de twee game engines, wat waardevolle inzichten zal leveren voor toekomstige projecten in gameontwikkeling.


%---------- Verwachte resultaten ----------------------------------------------
\section{Verwacht resultaat, conclusie}%
\label{sec:verwachte_resultaten}

De gebruikersinterface van Godot onderscheidt zich door haar eenvoud en intuïtieve design, wat in het bijzonder gunstig lijkt voor beginnende ontwikkelaars die zich richten op 2D-platformers. Godot's specialisatie in dit genre belooft een soepelere leercurve. Aan de andere kant staat Unity bekend om zijn robuustheid en uitgebreide bronnen, dankzij een langere aanwezigheid in de industrie. Deze factoren maken Unity een aantrekkelijke optie voor complexere projecten. Echter, voor aspirant-ontwikkelaars zonder eerdere ervaring in game-ontwikkeling lijkt Godot de meer toegankelijke keuze te zijn.


